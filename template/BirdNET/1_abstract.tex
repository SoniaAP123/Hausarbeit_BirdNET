\section{Abstract}

%Quelle : 
https://nbn-resolving.org/urn:nbn:de:bsz:ch1-qucosa2-369869

%so in der Art einen Abstract schreiben plus erwähnen, dass dieser Arbeit als Ziel hat 24h laufen zu können und somit die ursprümglihce BirdNET App erweitert wurde

Die automatisierte Überwachung der Vogelstimmenaktivität und der Artenvielfalt kann ein revolutionäres Werkzeug für Ornithologen, Naturschützer und Vogelbeobachter sein, um bei der langfristigen Überwachung kritischer Umweltnischen zu helfen. Tiefe künstliche neuronale Netzwerke haben die traditionellen Klassifikatoren im Bereich der visuellen Erkennung und akustische Ereignisklassifizierung übertroffen. Dennoch erfordern tiefe neuronale Netze Expertenwissen, um leistungsstarke Modelle zu entwickeln, trainieren und testen. Mit dieser Einschränkung und unter Berücksichtigung der Anforderungen zukünftiger Anwendungen wurde eine umfangreiche Forschungsplattform zur automatisierten Überwachung der Vogelaktivität entwickelt: BirdNET. Das daraus resultierende Benchmark-System liefert state-of-the-art Ergebnisse in verschiedenen akustischen Bereichen und wurde verwendet, um Expertenwerkzeuge und öffentliche Demonstratoren zu entwickeln, die dazu beitragen können, die Demokratisierung des wissenschaftlichen Fortschritts und zukünftige Naturschutzbemühungen voranzutreiben.