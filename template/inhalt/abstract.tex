\section*{Abstract}


Ornithologen bzw. Vogelkundler nutzen das Monitoring von Vögeln für den Vogeltierschutz.

Monitoring ist die dauerhafte Überwachung von bestimmten Systemen.

Im Bereich der Biologie findet sie Anwendung bei Tier- und Pflanzenarten für die Erfassung der Bestände. Es ist durch solche Beobachtungen möglich, Pflanzen und Tiere zu erforschen. Die erfassten Monitoring-Daten helfen Biologen und Umweltschützern den Arten- und Lebensraum nachzuvollziehen und zu schützen .

Diese Arbeit stellt ein KI-gestütztes Tool vor, welches das Monitoring von Vögeln automatisiert unterstützt.

Über ein Mikrofon kann ein Computer mindestens 24 Stunden lang kontinuierlich die Umgebung aufnehmen. Die Software analysiert die Audioaufnahmen basierend auf einem Netzwerk, welches auf die Erkennung von Vogelarten anhand des Vogelgesangs trainiert ist. Anschließend selektiert die Software aus den Audioaufnahmen einzelne Abschnitte und isoliert daraus den erkannten Vogelgesang.

 

%Der Webbrowser?


%https://vogelmonitoring-rlp.de/portfolio/monitoring-wofuer

% Quelle: https://de.wikipedia.org/wiki/Ornithologie


