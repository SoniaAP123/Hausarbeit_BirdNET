\section{Einleitende Übersicht%
         \label{sec:Einleitung}}

Thema dieses Projekts ist die Entwicklung einer Software zur praktischen Anwendung in der Erkennung von Vogelarten anhand ihres Gesangs.

Das Projekt soll in der Lage sein, ununterbrochen 24 stundenlang Vogelstimmen in der freien Natur aufzunehmen und zu analysieren. Dieses Verfahren ermöglicht es Ornithologen (aktive Menschen im Bereich Vogelkunde) verschiedene Informationen im Bereich Ökologie und Taxonomie zu erlangen. Das Projekt ist ein Hilfsmittel für spezielle Untersuchungen. Diese Untersuchungen können das Monitoring

\begin{itemize}

\item von den jeweiligen verteilten Populationen

\item des Vogelzugs

\item der versprengten Arten

\end{itemize}

sein.\bigskip

Das Programm soll folgende Kriterien erfüllen:

\begin{itemize}

\item Starten des Programms per Remote-Zugriff

\item Aufnahme und Analyse laufen mindestens 24 Stunden ununterbrochen parallel

\item Aufnahmen, die Vogelgesang beinhalten, für statistische Auswertungen wie Vogelzählung separat speichern und auswerten

\end{itemize}
\bigskip

Die vorliegende Arbeit ist folgendermaßen aufgebaut:

\begin{enumerate}

\item Abschnitt \ref{sec:Konzept} stellt das Konzept dar

\item Abschnitt \ref{sec:versuchsaufbau} skizziert und erklärt den theoretischen Versuchsaufbau

\item Abschnitt \ref{sec:umsetzung} erklärt die Umsetzung (Hardware/Software)

\item Abschnitt \ref{sec:versuchsdurchfuehrung} demonstriert den durchgeführten Versuch

\item Abschnitt ... validiert die Ergebnisse aus den Versuchsdurchführungen %soll es dazu einen extra ABschnitt geben?

\item in Abschnitt \ref{sec:zusammen} sind die Ergebnisse validiert und gibt einen Ausblick auf geschaffene Möglichkeiten durch dieses Projekt sowie mögliche weitere Entwicklungen des Projektes

\end{enumerate}

 


% https://de.wikipedia.org/wiki/Taxonomie

 %https://www.nabu.de/natur-und-landschaft/naturschutz/weltweit/naturschutzprojekte/afribirds/index.html
%https://www.nabu.de/natur-und-landschaft/naturschutz/weltweit/naturschutzprojekte/afribirds/27164.html


